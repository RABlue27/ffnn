\documentclass[12pt]{article}
\usepackage{amsmath}
\usepackage{graphicx}
\usepackage{caption}
\usepackage{subcaption}
\usepackage{booktabs}
\usepackage{float}
\usepackage[utf8]{inputenc}
\usepackage{geometry}
\usepackage{multirow}
\usepackage{setspace}
\usepackage{parskip}
\usepackage{svg}
\usepackage[bottom]{footmisc}
\usepackage{tikz}
\usepackage[section]{placeins}
\usepackage{listings}
\documentclass{article}
\usepackage{listings}
\usepackage{xcolor}

\definecolor{codegreen}{rgb}{0,0.6,0}
\definecolor{codegray}{rgb}{0.5,0.5,0.5}
\definecolor{codepurple}{rgb}{0.58,0,0.82}
\definecolor{backcolour}{rgb}{0.95,0.95,0.92}

\lstdefinestyle{mystyle}{
    backgroundcolor=\color{backcolour},   
    commentstyle=\color{codegreen},
    keywordstyle=\color{magenta},
    numberstyle=\tiny\color{codegray},
    stringstyle=\color{codepurple},
    basicstyle=\ttfamily\footnotesize,
    breakatwhitespace=false,         
    breaklines=true,                 
    captionpos=b,                    
    keepspaces=true,                 
    numbers=left,                    
    numbersep=5pt,                  
    showspaces=false,                
    showstringspaces=false,
    showtabs=false,                  
    tabsize=2
}

\lstset{style=mystyle}
% This style is used to create block diagrams, you'll find it useful since many of your figures would be of that form, I'll try add more styles in the future :)
\usetikzlibrary{trees,positioning,fit,calc}
\tikzset{block/.style = {draw, fill=blue!20, rectangle,
                         minimum height=3em, minimum width=4em},
        input/.style = {coordinate},
        output/.style = {coordinate}
}

\usepackage[section]{minted}
\usepackage{xcolor}
\usemintedstyle{porland}

\usepackage{chngcntr}
\counterwithin{figure}{section}

\renewcommand{\arraystretch}{1.5}

\usepackage[hidelinks]{hyperref}
\hypersetup{
    linktoc=all
}

\renewcommand\listingscaption{Listing}
\renewcommand\listoflistingscaption{List of Listings}

\usepackage{scrhack}
\usepackage{tocbasic}
\setuptoc{lol}{levelup}

\usepackage{indentfirst}
\geometry{a4paper, margin=1in}

%----------EDIT COVER INFO HERE -----------------%

\def \LOGOPATH {assets/birzeit-logo.svg}
\def \DEPARTEMENT {COSC 4P80}
\def \COURSENAME {Assign 2}
\def \REPORTTITLE {Feed Forward Network}
\def \STUDENTNAME {Alex Duclos}
\def \STUDENTID {6738884}
\def \INSTRUCTOR {Dave Bockus}

%------------------------------------------------%

\setlength{\parindent}{0em}
\setlength{\parskip}{0em}

\begin{document}

\pagenumbering{Roman}

\begin{titlepage}
    \vfill
    \begin{center}
        \includesvg[width=0.7\textwidth]{\LOGOPATH} \\
        \hfill \\
        \Large{\DEPARTEMENT} \\
        \Large{\COURSENUM\;-\;\COURSENAME} \\
        \vfill
        \textbf{\LARGE{\REPORTTITLE}}
    \end{center}
    \vfill
    \begin{flushleft}
        \Large{\textbf{Prepared by:} \STUDENTNAME} \\
        \Large{\textbf{Instructor:} \INSTRUCTOR} \\
        \Large{\textbf{Date:} \today}
    \end{flushleft}
    \vfill
\end{titlepage}

%--------------ABSTRACT ------------------------%
{
\centering
\section*{Abstract}
Developing a neural network to figure out if a electric motor is malfunctioning based on behaviour. Using a Fourier transform we create a multi-layer neural network to distinguish data. Using back propagation to learn. 
\clearpage
}

%-----------------------------------------------%

\tableofcontents
\clearpage

\setlength{\parskip}{\baselineskip}%

\pagenumbering{arabic}

%--------------INTRODUCTION ---------------------%

\section{Setup}
First I created a neural network which can be adapted based on user inputs. Allowing user to specify the hidden layer count and to automatically create a input layer based on the length of the input. I created the initial weights using a random distribution function. The network has two functions for data, one forward which sends the data forward and returns the expected output, which is 0 or 1, as well as a backwards function which calculates the error and modifies the weights to try to correct this error. 

We can train this network by using these functions in tandem to identify what output the network gives for an input, calculating the error off of it, and back-propagating for the entire data set for every epoch. 

We used a sigmoid activation function as well as the numpy library to handle the math behind the network, as well as speeding up all operations as the underlying code for numpy is written in C. We used matplot lib to form the graphs.

We also normalize out inputs. The network also dynamically creates the input layer size based off of the inputted file. 

\section{Supporting Code}
\begin{lstlisting}[language=Python, caption=Python example]
import numpy as np
    
def sigmoid(x):
    return 1 / (1 + np.exp(-x))

def sigmoid_derivative(x):
    return x * (1 - x)

def backward(self, X, y, output, learning_rate):
    # Compute the difference between the output and the true label
    output_error = y - output
    
    # Compute the derivative of the output layer and update weights
    output_delta = output_error * sigmoid_derivative(output)
    self.weights2 += learning_rate * np.dot(self.hidden.T, output_delta)
    
    # Compute the derivative of the hidden layer and update weights
    hidden_error = np.dot(output_delta, self.weights2.T)
    hidden_delta = hidden_error * sigmoid_derivative(self.hidden)
    self.weights1 += learning_rate * np.dot(X.T, hidden_delta)
    
def train(self, X, y, num_epochs, learning_rate):
    for i in range(num_epochs):
        # Compute the output of the network for the current input
        output = self.forward(X)

        # Compute the mean squared error of the output
        error = np.mean((output - y) ** 2)
    
        # Update the weights based on the output and true labels
        self.backward(X, y, output, learning_rate)

def forward(self, X):
    # Compute the dot product of the input with the first set of weights, and apply the sigmoid function
    self.hidden = sigmoid(np.dot(X, self.weights1))
    
    # Compute the dot product of the hidden layer with the second set of weights, and apply the sigmoid function
    output = sigmoid(np.dot(self.hidden, self.weights2))
    return output
        
\end{lstlisting}

\subsection{Code Explained}
These are the primary functions in the neural network. Sigmoid and Sigmoid derivative are used in backpropegation and backward and train are used to initalize the matrix. 

Calling nn.foward() will allow you to predict an output, which is used both for testing, training, and actually getting usable results out of the network. 

\section{Discussion Part 1}
I used a static learning rate of 0.1 The MSE over 10000 epoch is as follow:
\begin{table}[h!]
  \centering
  \caption{MSE per Epoch}
  \label{tab:mse}
  \begin{tabular}{|c|c|}
    \hline
    \textbf{Epoch} & \textbf{MSE} \\
    \hline
    0 & 0.28015124592765495 \\
    \hline
    1000 & 0.015238976324625628 \\
    \hline
    2000 & 0.0053447927813529954 \\
    \hline
    3000 & 0.002537586348674161 \\
    \hline
    4000 & 0.0015096883581622543 \\
    \hline
    5000 & 0.00102516931339903 \\
    \hline
    6000 & 0.0007560355485122358 \\
    \hline
    7000 & 0.0005892362708036518 \\
    \hline
    8000 & 0.00047758474800363253 \\
    \hline
    9000 & 0.00039849267690033376 \\
    \hline
  \end{tabular}
\end{table}


We can see that it starts with a .28 and quickly goes down to .02 after only 1000 epochs. More epochs gets a more specific network. This was on the 1000 sample size training data with 12 hidden nodes. 

Using 6 hidden nodes with 150 sized training data, we get 

\begin{table}[h]
\centering
\begin{tabular}{|c|c|}
\hline
Epoch & MSE \\
\hline
0 & 0.2973518896461779 \\
1000 & 0.08006745135601322 \\
2000 & 0.05881557965133051 \\
3000 & 0.044334162834307696 \\
4000 & 0.03531914796699035 \\
5000 & 0.029991267475922973 \\
6000 & 0.02678223126552964 \\
7000 & 0.024762332351237735 \\
8000 & 0.023430424622077842 \\
9000 & 0.02251373317143947 \\
\hline
\end{tabular}
\end{table}

We can see here that the optimal amount of MSE is 0.02. So the ideal and most specified input for our network is dependent both on input size as well as hidden layer size.

Furthermore, we implement a  test network function, which allows us to test an input based off a neural network as an input. 

\section{Discussion Part 2}
Using the 150 sized file, and 10,000 epochs we generate this graph [Figure 1].

\begin{figure}[p]
  \centering
  \includegraphics[width=0.5\textwidth]{Figure_1.png}
  \caption{150 input, 10000 epochs, 8}
  \label{fig:my_label}
\end{figure}

We can see that all inputs trend together.

When using 25 as a hidden size for the same input, we can generate [Figure 2]. We can see in image 2 a much quicker conversion, taking less than half as many epochs for all the training sets to converge. 

\begin{figure}[h]
  \centering
  \includegraphics[width=0.5\textwidth]{Figure_2.png}
  \caption{150 input, 10000 epochs, 25 hidden}
  \label{fig:my_label}
\end{figure}


\section{Discussion Part 3}
Implementing momentum very quickly allows us to converge. I used an initial value of .9 and multiplied it by .9 for every epoch as long as it remained above 0.5. Continuing to use the 150 input training set with 8 hidden nodes, we create this data-set using the same splitting as in Part C. 

\begin{table}[h]
    \centering
    \begin{tabular}{c|c}
        \hline
        \textbf{Epoch} & \textbf{MSE} \\
        \hline
        0 & 0.6738298155661789 \\
        5 & 0.03979317883930451 \\
        10 & 0.014718938398791081 \\
        15 & 0.009017340839524442 \\
        20 & 0.Epoch 20, MSE: 0.006453993235001123
 \\
        \hline
    \end{tabular}
    \caption{MSE values for the neural network with a learning rate of 0.9.}
    \label{table:neural_network}
\end{table}

\begin{table}[h]
    \centering
    \begin{tabular}{c|c}
        \hline
        \textbf{Epoch} & \textbf{MSE} \\
        \hline
        0 & 0.7131921298303292 \\
        5 & 0.01641538707289942 \\
        10 & 0.007956890088508728 \\
        15 & 0.005290668342611634 \\
        20 & 0.003953661542173633 \\
        \hline
    \end{tabular}
    \caption{MSE values for the neural network with a learning rate of 0.9.}
    \label{table:neural_network}
\end{table}


As we can see from these tables, it only takes fewer epochs to generate 
a MSE which is low compared to the previous parts. A graph is [Figure 3]

\begin{figure}[h]
  \centering
  \includegraphics[width=0.5\textwidth]{Figure_3.png}
  \caption{Momentum Race.}
  \label{fig:my_label}
\end{figure}

\section{Discussion Part 4}
For this part we used 150 input size again, with 8 hidden inputs. We'll use 100 epochs for this portion. 

\begin{table}[htbp!]
\centering
\caption{MSE for different training and test sets}
\label{tab:mse}
\begin{tabular}{|c|c|c|}
\hline
\textbf{Training Set} & \textbf{Test Set} & \textbf{MSE} \\ \hline
AB & B & 0.5101357447050856 \\ 
AB & A & 0.5101704637535027 \\ 
AB & C & 0.0003136482533078072 \\ \hline
AC & B & 0.5113393051410082 \\ 
AC & A & 0.5114976716671992 \\ 
AC & C & 0.0005603099963618362 \\ \hline
BC & B & 0.45984055367859344 \\ 
BC & A & 0.45525311189230727 \\ 
BC & C & 0.9687893598238666 \\ \hline
AB & B & 0.5006573308965951 \\ 
AB & A & 0.5006207892715613 \\ 
AB & C & 0.0005676994714459097 \\ \hline
AC & B & 0.5060711660774762 \\ 
AC & A & 0.5063464172999042 \\ 
AC & C & 0.0008245561459402124 \\ \hline
BC & B & 0.45193490601748487 \\ 
BC & A & 0.4502200094671267 \\ 
BC & C & 0.9547093783465248 \\ \hline
\end{tabular}
\end{table}

Looking at this table, we can see some very obvious things. Training set A and B always cause a high MSE. This is due to the uneven nature of the data set.  Furthermore, training set BC generated a terrible result and even a MSE near 1. 

The best results occur when using AX and C as our test set.

\section{Discussion Part 5}
In this section we expand the neural network to have 3 hidden layers. This remains dynamic like the previous parts, where the user can select the number of nodes for each layer. The fundamentals of the neural network don't change here either, back propagating from layer N to layer N-1, and forward feeding from layer N to N+1.

In this section we will use 6 hidden nodes on each hidden layer.

When using only 10 epochs, a number which was completely fine in the previous part 

\begin{tabular}{|c|c|c|}
\hline
Training Set & Test Set & MSE \\ \hline
AB & B & 0.481025264706865 \\
AB & A & 0.48100277044854295 \\
AB & C & 0.0021487104532940562 \\
\hline
AC & B & 0.44933642333026874 \\
AC & A & 0.4493429071394288 \\
AC & C & 0.006501194178445878 \\
\hline
BC & B & 0.4340856940162583 \\
BC & A & 0.43387600626338024 \\
BC & C & 0.9170176233876534 \\
\hline
AB & B & 0.448339305168056 \\
AB & A & 0.44825906440435737 \\
AB & C & 0.006548612993577 \\
\hline
AC & B & 0.4450806607045077 \\
AC & A & 0.44518982002687735 \\
AC & C & 0.007314290323871856 \\
\hline
BC & B & 0.43868399271401676 \\
BC & A & 0.4382040096951598 \\
BC & C & 0.9269480701668228 \\
\hline
\end{tabular}

The MSE of these charts are at best equal, but in many cases worse. If we expand to 1000 epochs, however, we get a table such as this.

\begin{tabular}{|c|c|c|}
\hline
Training Set & Test Set & MSE \\
\hline
AB & B & 0.5258119509902481 \\
AB & A & 0.5258119371619241 \\
AB & C & 3.4692064770355837e-06 \\
\hline
AC & B & 0.5260299458357827 \\
AC & A & 0.5260299669051591 \\
AC & C & 2.7425902893942473e-06 \\
\hline
BC & B & 0.470684726866821 \\
BC & A & 0.470636053854254 \\
BC & C & 0.9966896264127715 \\
\hline
AB & B & 0.5253575575532294 \\
AB & A & 0.5253575200755032 \\
AB & C & 5.257848448062897e-06 \\
\hline
AC & B & 0.5255183242405246 \\
AC & A & 0.5255189749693983 \\
AC & C & 5.651470562569696e-06 \\
\hline
BC & B & 0.47035317543555066 \\
BC & A & 0.4703469572559509 \\
BC & C & 0.9960309729556156 \\
\hline
\end{tabular}

In this table we can see that worst case scenario remains the same, but the MSE for our best case scenario rapidly drops, and since we aren't using our test set for training in these questions, we can be reasonably certain that we're not memorizing but instead doing a fantastic job of predicting. 


\section{Conclusion}
Overall, this project required a fundamental understanding of FF neural networks. Implementing back propagation, and creating a flexible network were both required to succeed. Using sigmoid and numpy, we generated a quick and efficient network to generate consistent outputs. 

Our results also show an importance in test set data. Multiple layers added to the network allow for a well generalized network, Splitting the data up gave a drastic difference in the effectiveness of the training. 

\end{document}
\clearpage

\bibliographystyle{IEEEtran}
\bibliography{IEEEabrv,cites}

\clearpage

\listoffigures

\clearpage

\listoftables

\clearpage

\listoflistings

\end{document}